% -*-latex-*-

\chapter{Results and Conclusions}
\label{C8}

In this chapter, preliminary results for the proton asymmetries and polarized cross-section differences are presented. The spin structure functions and their contributions to the spin polarizabilities are discussed as well. Since the analysis is not finalized for E08-027, only the L-HRS data with 1.710, 2.253 and 3.350 GeV beam energies are analyzed in this thesis.

\section{Asymmetry Results}
\label{C8S1}

\Cref{C7S1E3,C7S1E4} can be used to extract the physics asymmetry. The beam current has been discussed in \Cref{C5S2SS2} and the livetime correction has been discussed in \Cref{C5S4SS2}. The beam polarization and target polarization has been discussed in \Cref{C5S2SS4} and \Cref{C5S3SS3}, respectively. The preliminary dilution factors have been given in \Cref{C7S4}. Thus, the physics asymmetries can be extracted. The results are shown in \Cref{C8S1F1,C8S1F2,C8S1F3,C8S1F4}.

\begin{figure}[p!]
  \centering
  \begin{subfigure}[t]{0.79\textwidth}
    \includegraphics[width=\textwidth]{figs/asymmetry-22535000.pdf}
    \caption{Longitudinal configuration. \label{C8S1F1a}}
  \end{subfigure}
  \begin{subfigure}[t]{0.79\textwidth}
    \includegraphics[width=\textwidth]{figs/asymmetry-22535090.pdf}
    \caption{Transverse configuration. \label{C8S1F1b}}
  \end{subfigure}
  \caption[Physics asymmetries with $E=2.253$ GeV and $B=5.0$ T.]{Physics asymmetries for the configurations with 2.253 GeV beam energy and 5.0 T target field. The uncertainties shown are only statistical. \label{C8S1F1}}
\end{figure}

\begin{figure}[p!]
  \centering
  \includegraphics[width=0.79\textwidth]{figs/asymmetry-17102590.pdf}
  \caption[Physics asymmetries with $E=1.710$ GeV and $B=2.5$ T.]{Physics asymmetries for the configurations with 1.710 GeV beam energy and 2.5 T transverse target field. The uncertainties shown are only statistical. \label{C8S1F2}}
\end{figure}

\begin{figure}[p!]
  \centering
  \includegraphics[width=0.79\textwidth]{figs/asymmetry-22532590.pdf}
  \caption[Physics asymmetries with $E=2.253$ GeV and $B=2.5$ T.]{Physics asymmetries for the configurations with 2.253 GeV beam energy and 2.5 T transverse target field. The uncertainties shown are only statistical. \label{C8S1F3}}
\end{figure}

\begin{figure}[tb!]
  \centering
  \includegraphics[width=0.79\textwidth]{figs/asymmetry-33505090.pdf}
  \caption[Physics asymmetries with $E=3.350$ GeV and $B=5.0$ T.]{Physics asymmetries for the configurations with 3.350 GeV beam energy and 5.0 T transverse target field. The uncertainties shown are only statistical. \label{C8S1F4}}
\end{figure}

The statistical uncertainties of the physics asymmetries are shown in the plots. If we assume the total event amount is $N\approx2N^+\approx2N^-$, the absolute statistical uncertainty of the asymmetries is $\sim1/\sqrt{N}$. This statement is valid because the fluctuations of the event amount $N^{\pm}$ follow the Poisson distribution, which are $\sqrt{N^{\pm}}$ here. However, the DAQ event rate is reduced by applying a prescale factor $ps$ when the raw trigger rate is high. In this case, the fluctuations of the event amount no longer follow the Poisson distribution, and need to be corrected by a factor $S$ \cite{Qiang2007}:
\begin{equation} \label{C8S1E1}
S = \sqrt{1-LT\cdot f_A(1-\frac{1}{ps})},
\end{equation}
where $LT$ is the livetime correction of the DAQ system and $f_A$ is the acceptance correction which is defined as $f_A=N_{\mathrm{accepted}}/N_{\mathrm{total}}$. The statistical uncertainty can then be written as:
\begin{equation} \label{C8S1E2}
\delta A \simeq \frac{1}{2}\sqrt{\frac{S_+^2}{N_+}+\frac{S_-^2}{N_-}}.
\end{equation}

During the experiment, the data is taken in several ``runs'', each containing about 7 million events. \cref{C8S1E2} can be used to calculate the uncertainty for each run, and the final asymmetry must be combined using a statistically weighted average:
\begin{equation} \label{C8S1E3}
A = \frac{\sum_iA_i/\delta A_i^2}{\sum_i1/\delta A_i^2},
\end{equation}
\begin{equation} \label{C8S1E4}
\delta A = \sqrt{\frac{1}{\sum_i1/\delta A_i^2}},
\end{equation}
where $A_i$ is the asymmetry calculated for each run, and $\delta A_i$ is the statistical uncertainty given by \cref{C8S1E2}.

\section{Radiative Correction}
\label{C8S2}

The Feynman diagram shown in \Cref{C2S1F1} only considers the leading order process, which is known as the Born approximation. This is assumed for theoretical analyses of the lepton-nucleon scattering. However, the data contains all of the high order effects which need to be corrected if the data was compared with the theoretical results. This correction is referred to as the radiative correction.

The radiative correction arises from several different sources. The virtual photon one-loop corrections are shown in \Cref{C8S2F1}. It includes (a) the vacuum polarization correction where the virtual photon splits into an $e^-/e^+$ pair and acts as an electric dipole, (b) the vertex correction, (c)(d) the electron self-energy which contribute to the renormalization of the electron mass and (e)(f) the Bremsstrahlung radiation.

\begin{figure}[tb!]
  \centering
  \includegraphics[width=0.75\textwidth]{figs/one-loop-corrections.png}
  \caption[Diagrams for next-to-leading order corrections.]{Diagrams for next-to-leading order corrections. Plot reproduced from \cite{Zielinski2014b}. \label{C8S2F1}}
\end{figure}

Diagrams (a), (b), (c) and (d) are considered to be relatively small compared to the contributions of the internal and external Bremsstrahlung. The internal Bremsstrahlung happens when the electrons interacting with the nucleon, whereas the external Bremsstrahlung happens when the electrons passes through the materials before or after the interaction. In addition to the Bremsstrahlung, energy can be lost when an electron passes through the materials. It is mainly due to the ionization effect and is dependent on the radiation thickness of the materials the electron passes through. The ionization energy loss is typically in the order of a few MeVs.

The radiative corrections should be applied to the asymmetry and the cross-section results extracted from the data. For preliminary study, the results from data are not radiative corrected, but they are compared with some radiative corrected model predictions. The Mainz online partial-wave analysis of meson electroproduction (MAID) \cite{Drechsel2007} is used to generate the polarized cross-section differences. The radiative effects are always separated into the internal part and the external part for convenience. The POLRAD formalism \cite{Akushevich1997} is used to calculate the internal radiative effects. And the methods in Ref. \cite{Mo1969} developed by Mo and Tsai is used to describe the external radiative effects.

The fits of P. Bosted to the inclusive inelastic electron scattering \cite{Bosted2008} are used to generate the unpolarized cross-sections, which also need to be corrected by the radiative effects. The radiative effects are calculated using the same formalism as the polarized cross-section differences with the fits of P. Bosted as input for both the internal and external corrections.

\begin{figure}[p!]
  \centering
  \begin{subfigure}[t]{0.79\textwidth}
    \includegraphics[width=\textwidth]{figs/asymmetry-model-22535000.pdf}
    \caption{Longitudinal configuration. \label{C8S2F2a}}
  \end{subfigure}
  \begin{subfigure}[t]{0.79\textwidth}
    \includegraphics[width=\textwidth]{figs/asymmetry-model-22535090.pdf}
    \caption{Transverse configuration. \label{C8S2F2b}}
  \end{subfigure}
  \caption[Comparison of the radiated and unradiated model predictions.]{Comparison of the radiated and unradiated model predictions for the asymmetries of the kinematic setting with 2.253 GeV beam energy and 5.0 T target field. \label{C8S2F2}}
\end{figure}

\begin{figure}[p!]
  \centering
  \begin{subfigure}[t]{0.79\textwidth}
    \includegraphics[width=\textwidth]{figs/asymmetry-data-model-22535000.pdf}
    \caption{Longitudinal configuration. \label{C8S2F3a}}
  \end{subfigure}
  \begin{subfigure}[t]{0.79\textwidth}
    \includegraphics[width=\textwidth]{figs/asymmetry-data-model-22535090.pdf}
    \caption{Transverse configuration. \label{C8S2F3b}}
  \end{subfigure}
  \caption[Asymmetries with $E=2.253$ GeV and $B=5.0$ T.]{Comparison of the radiated model predictions with measured asymmetries for the kinematic settings with 2.253 GeV beam energy and 5.0 T target field (longitudinal and transverse configurations). \label{C8S2F3}}
\end{figure}

\begin{figure}[p!]
  \centering
  \includegraphics[width=0.79\textwidth]{figs/asymmetry-data-model-17102590.pdf}
  \caption[Asymmetries with $E=1.710$ GeV and $B=2.5$ T.]{Comparison of the radiated model predictions with measured asymmetries for the kinematic settings with 1.710 GeV beam energy and 2.5 T transverse target field. \label{C8S2F4}}
\end{figure}

\begin{figure}[p!]
  \centering
  \includegraphics[width=0.79\textwidth]{figs/asymmetry-data-model-22532590.pdf}
  \caption[Asymmetries with $E=2.253$ GeV and $B=2.5$ T.]{Comparison of the radiated model predictions with measured asymmetries for the kinematic settings with 2.253 GeV beam energy and 2.5 T transverse target field. \label{C8S2F5}}
\end{figure}

\begin{figure}[tb!]
  \centering
  \includegraphics[width=0.79\textwidth]{figs/asymmetry-data-model-33505090.pdf}
  \caption[Asymmetries with $E=3.350$ GeV and $B=5.0$ T.]{Comparison of the radiated model predictions with measured asymmetries for the kinematic settings with 3.350 GeV beam energy and 5.0 T transverse target field. \label{C8S2F6}}
\end{figure}

The elastic tail must also be considered since it becomes significant in the resonance region. The MASCARD code \cite{Afanasev2001} is used to generate polarized elastic cross-sections. And the form factors from Mo and Tsai are used to calculate the radiative effects for both the polarized and unpolarized radiative effects.

The radiated and unradiated model predictions for the asymmetries is compared in \Cref{C8S2F2}. Here only the two kinematic settings (longitudinal and transverse configurations) with 2.253 GeV beam energy and 5.0 T target field are shown as examples. The invariant mass $W$ and the four-momentum transfer $Q$ are used in the models as kinematic inputs. The data is used to fit the relation between the scattering angle and $W$ for each kinematic setting, and $Q$ can be determined with $W$ and the scattering angle. \Cref{C8S2F7} shows the relations between $W$ and the scattering angle. Again only the two settings with 2.253 GeV beam energy and 5.0 T target field are shown in the figure as examples. The variation of the scattering angle is an effect of the target magnetic field. Thus, the variation is very small for the longitudinal configuration but for transverse configuration it is significant.

\begin{figure}[tb!]
  \centering
  \includegraphics[width=0.6\textwidth]{figs/scattering-angle.pdf}
  \caption[Relations between the scattering angle and $W$.]{Relations between the scattering angle and $W$ for the kinematic settings with 2.253 GeV beam energy and 5.0 T target field (longitudinal and transverse configuration). \label{C8S2F7}}
\end{figure}

\Cref{C8S2F3,C8S2F4,C8S2F5,C8S2F6} shows the comparison of the radiated model predictions with the physics asymmetries extracted in the previous section. The uncertainty of the model prediction arises from several different sources. The fits of P. Bosted contribute a relative uncertainty of 5\% \cite{Bosted2008} and the Mo and Tsai formalism contributes a relative uncertainty of 4\% \cite{Mo1969}. The MAID group does not provide the fit uncertainties since the fit uncertainty is unrealistically small in this case due to the large number of data points included in the fit. Thus, we will use the difference between the model prediction and our data as the uncertainty of the model when we calculate the cross-section differences in the next section. The acceptance effects also contribute to the uncertainty when we fit the relations between $W$ and the scattering angle with data. Since the acceptance analysis is still on-going, its contribution to the uncertainty of the model prediction is not determined yet.

\section{Polarized Cross-Section Differences}
\label{C8S3}

\begin{figure}[p!]
  \centering
  \begin{subfigure}[t]{0.79\textwidth}
    \includegraphics[width=\textwidth]{figs/xsdiff-model-22535000.pdf}
    \caption{Longitudinal configuration. \label{C8S3F1a}}
  \end{subfigure}
  \begin{subfigure}[t]{0.79\textwidth}
    \includegraphics[width=\textwidth]{figs/xsdiff-model-22535090.pdf}
    \caption{Transverse configuration. \label{C8S3F1b}}
  \end{subfigure}
  \caption[Cross-section differences with $E=2.253$ GeV and $B=5.0$ T.]{Comparison of the radiative corrected and uncorrected cross-section differences for the kinematic settings with 2.253 GeV beam energy and 5.0 T target field (longitudinal and transverse configurations). \label{C8S3F1}}
\end{figure}

\begin{figure}[p!]
  \centering
  \includegraphics[width=0.79\textwidth]{figs/xsdiff-model-17102590.pdf}
  \caption[Cross-section differences with $E=1.710$ GeV and $B=2.5$ T.]{Comparison of the radiative corrected and uncorrected cross-section differences for the kinematic settings with 1.710 GeV beam energy and 2.5 T transverse target field. \label{C8S3F2}}
\end{figure}

\begin{figure}[p!]
  \centering
  \includegraphics[width=0.79\textwidth]{figs/xsdiff-model-22532590.pdf}
  \caption[Cross-section differences with $E=2.253$ GeV and $B=2.5$ T.]{Comparison of the radiative corrected and uncorrected cross-section differences for the kinematic settings with 2.253 GeV beam energy and 2.5 T transverse target field. \label{C8S3F3}}
\end{figure}

\begin{figure}[tb!]
  \centering
  \includegraphics[width=0.79\textwidth]{figs/xsdiff-model-33505090.pdf}
  \caption[Cross-section differences with $E=3.350$ GeV and $B=5.0$ T.]{Comparison of the radiative corrected and uncorrected cross-section differences for the kinematic settings with 3.350 GeV beam energy and 5.0 T transverse target field. \label{C8S3F4}}
\end{figure}

The polarized cross-section differences can be calculated via \Cref{C7S1E9}. The asymmetries $A_{\parallel,\perp}$ has been discussed in previous sections. Since the acceptance analysis is still on-going, the unpolarized cross-sections extracted from our data are not reliable yet. Thus, the fits of P. Bosted \cite{Bosted2008} are used as the unpolarized cross-section to extract $\Delta\sigma_{\parallel,\perp}$.

In order to extract the polarized structure functions, the cross-section differences need to be radiative corrected. The standard method to perform radiative correction is to deconvolve the spectrum extracted from the data with the help of simulation. In this thesis, we will perform the radiative correction to the asymmetries in an alternative way. If we denote the radiated and unradiated model predictions of asymmetries by $A_{\mathrm{rad}}^{\mathrm{model}}$ and$A_{\mathrm{unrad}}^{\mathrm{model}}$ respectively, the differences between the radiated and unradiated models can be expressed as:
\begin{equation} \label{C8S3E1}
\Delta_{\mathrm{RC}}^{\mathrm{model}} = A_{\mathrm{unrad}}^{\mathrm{model}}+A_{\mathrm{rad}}^{\mathrm{model}}.
\end{equation}
$\Delta_{\mathrm{RC}}^{\mathrm{model}}$ is taken as the radiative correction, and the asymmetries and the cross-section differences can be expressed as:
\begin{equation} \label{C8S3E2}
A_{\mathrm{corrected}} = A_{\mathrm{uncorrected}}-\Delta_{\mathrm{RC}}^{\mathrm{model}},
\end{equation}
and
\begin{equation} \label{C8S3E3}
\Delta\sigma_{\parallel,\perp}^{\mathrm{corrected}} = 2A_{\parallel,\perp}^{\mathrm{corrected}} \cdot \sigma_0.
\end{equation}

The radiative-corrected cross-section differences are shown in \Cref{C8S3F1,C8S3F2,C8S3F3,C8S3F4}. The radiated and unradiated model predictions of the cross-section differences are also shown in the figures for comparison. The error bars on each data point are only the statistical uncertainty.

The systematic uncertainty of the cross-section difference calculation has two major contributions: the systematic uncertainties of the asymmetries and the unpolarized cross-sections. Since the unpolarized cross-sections are given by the fits of P. Bosted, the systematic uncertainties contributed by $\sigma_0$ are 5\%. There are several different contributions to the systematic uncertainty of the asymmetry calculation. The dilution factors are calculated from the packing fraction with \cref{C7S4E11}. The uncertainties of the packing fractions are listed in \Cref{C7S3T1} and there is an additional 5\% uncertainty since the dilution factors are extracted from the P. Bosted's fits. The uncertainties of the beam polarization and the target polarization are $\sim$1.7\% and $\sim$1.2\% respectively, which has been described in \Cref{C5}. The largest contribution to the uncertainty comes from the models we used to perform the radiative correction. As mentioned in \Cref{C8S2}, the difference between the MAID model and our data is taken as the uncertainty in the analysis. Thus, we could combine all of these contributions to estimate the systematic uncertainty of the cross-section differences. The estimations of the systematic uncertainty are shown as the grey bands in \Cref{C8S3F1,C8S3F2,C8S3F3,C8S3F4}.

\section{\texorpdfstring{Spin Structure Function $g_2^p$}{Spin Structure Function g2p}}
\label{C8S4}

\begin{figure}[p!]
  \centering
  \begin{subfigure}[t]{0.79\textwidth}
    \includegraphics[width=\textwidth]{figs/g1g2-model-22535000.pdf}
    \caption{Longitudinal configuration. \label{C8S4F1a}}
  \end{subfigure}
  \begin{subfigure}[t]{0.79\textwidth}
    \includegraphics[width=\textwidth]{figs/g1g2-model-22535090.pdf}
    \caption{Transverse configuration. \label{C8S4F1b}}
  \end{subfigure}
  \caption[$g_1$ and $g_2$ results with $E=2.253$ GeV and $B=5.0$ T.]{$g_1$ and $g_2$ results for the kinematic settings with 2.253 GeV beam energy and 5.0 T target field (longitudinal and transverse configurations). The error bars on each data point are the statistical uncertainty. \label{C8S4F1}}
\end{figure}

\begin{figure}[p!]
  \centering
  \includegraphics[width=0.79\textwidth]{figs/g1g2-model-17102590.pdf}
  \caption[$g_2$ results with $E=1.710$ GeV and $B=2.5$ T.]{$g_2$ results for the kinematic settings with 1.710 GeV beam energy and 2.5 T target field. The error bars on each data point are the statistical uncertainty. \label{C8S4F2}}
\end{figure}

\begin{figure}[p!]
  \centering
  \includegraphics[width=0.79\textwidth]{figs/g1g2-model-22532590.pdf}
  \caption[$g_2$ results with $E=2.253$ GeV and $B=2.5$ T.]{$g_2$ results for the kinematic settings with 2.253 GeV beam energy and 2.5 T target field. The error bars on each data point are the statistical uncertainty. \label{C8S4F3}}
\end{figure}

\begin{figure}[tb!]
  \centering
  \includegraphics[width=0.79\textwidth]{figs/g1g2-model-33505090.pdf}
  \caption[$g_2$ results with $E=3.350$ GeV and $B=5.0$ T.]{$g_2$ results for the kinematic settings with 3.350 GeV beam energy and 5.0 T target field. The error bars on each data point are the statistical uncertainty. \label{C8S4F4}}
\end{figure}

In \Cref{C2S2}, we have derived the relations between the polarized cross-section differences and the polarized structure functions \cref{C2S2E25,C2S2E26}. Thus, the polarized structure functions $g_1$ and $g_2$ can be written in terms of the cross-section differences as:
\begin{align} \label{C8S4E1}
g_1 & = \frac{MQ^2}{4\alpha^2}\frac{y}{(1-y)(2-y)}\left[\Delta\sigma_\parallel+\tan\frac{\theta}{2}\Delta\sigma_\perp\right], \\ \label{C8S4E2}
g_2 & = \frac{MQ^2}{4\alpha^2}\frac{y^2}{2(1-y)(2-y)}\left[-\Delta\sigma_\parallel+\frac{1+(1-y)\cos\theta}{(1-y)\sin\theta}\Delta\sigma_\perp\right],
\end{align}
where $y=\nu/E$.

During E08-027, only the transverse cross-section differences $\Delta\sigma_\perp$ are measured in most cases. The model predictions of $\Delta\sigma_\parallel$ calculated in the previous section are used to extract the polarized structure functions. The model predictions will be replaced by the data of JLab EG4 experiment in the future work. For the kinematic setting with 2.253 GeV beam energy and 5.0 T target field, although both $\Delta\sigma_\perp$ and $\Delta\sigma_\parallel$ are measured during the experiment, the kinematics of the longitudinal and transverse configurations are not the same as shown in \Cref{C8S2F7}. Thus the model predictions are also used in these settings. The results for $g_1$ and $g_2$ are shown in \Cref{C8S4F1,C8S4F2,C8S4F3,C8S4F4}. The systematic uncertainties are shown as the grey bands in the plots.

\section{\texorpdfstring{Spin Polarizability $\dlt$}{Spin Polarizability delta\_\{LT\}}}
\label{C8S5}

\begin{figure}[p!]
  \centering
  \begin{subfigure}[t]{0.79\textwidth}
    \includegraphics[width=\textwidth]{figs/gamma0-model-22535000.pdf}
    \caption{Longitudinal configuration. \label{C8S5F1a}}
  \end{subfigure}
  \begin{subfigure}[t]{0.79\textwidth}
    \includegraphics[width=\textwidth]{figs/gamma0-model-22535090.pdf}
    \caption{Transverse configuration. \label{C8S5F1b}}
  \end{subfigure}
  \caption[$\gamma_0$ integrand with $E=2.253$ GeV and $B=5.0$ T.]{Value of the $\gamma_0$ integrand for the kinematic setting with 2.253 GeV beam energy and 5.0 T target field (longitudinal and transverse configurations). \label{C8S5F1}}
\end{figure}

\begin{figure}[p!]
  \centering
  \begin{subfigure}[t]{0.79\textwidth}
    \includegraphics[width=\textwidth]{figs/dlt-model-22535000.pdf}
    \caption{Longitudinal configuration. \label{C8S5F2a}}
  \end{subfigure}
  \begin{subfigure}[t]{0.79\textwidth}
    \includegraphics[width=\textwidth]{figs/dlt-model-22535090.pdf}
    \caption{Transverse configuration. \label{C8S5F2b}}
  \end{subfigure}
  \caption[$\dlt$ integrand with $E=2.253$ GeV and $B=5.0$ T.]{Value of the $\dlt$ integrand for the kinematic setting with 2.253 GeV beam energy and 5.0 T target field (longitudinal and transverse configurations).. \label{C8S5F2}}
\end{figure}

% \begin{figure}[p!]
%   \centering
%   \begin{subfigure}[t]{0.75\textwidth}
%     \includegraphics[width=\textwidth]{figs/gamma0-all.png}
%     \caption{Contribution to $\gamma_0$. \label{C8S5F3a}}
%   \end{subfigure}
%   \begin{subfigure}[t]{0.75\textwidth}
%     \includegraphics[width=\textwidth]{figs/dlt-all.png}
%     \caption{Contribution to $\dlt$. \label{C8S5F3b}}
%   \end{subfigure}
%   \caption[Contribution to the generalized spin polarizabilities from the resonance region.]{Contribution to the generalized spin polarizabilities from the resonance region. The unit is $10^{−4}$ fm${}^4$. \label{C8S5F3}}
% \end{figure}

As mentioned in \Cref{C4S2SS1}, the generalized polarizabilities $\gamma_0$ and $\dlt$ are moments of $g_1$ and $g_2$. With a preliminary result for the spin structure functions, we can begin to calculate the contributions of $g_1$ and $g_2$ to the polarizabilities. The integrands of $\gamma_0$ and $\dlt$ have already been shown in \cref{C4S2E8,C4S2E12}. The calculation result of the integrand of $\gamma_0$ is shown in \Cref{C8S5F1} and the value of the integrand of $\dlt$ is shown versus $x$ in \Cref{C8S5F2}. Here only the two kinematic settings (longitudinal and transverse configurations) with 2.253 GeV beam energy and 5.0 T target field are shown as examples.

%The contribution to the $\gamma_0$ and $\dlt$ integral from the resonance region is shown in \Cref{C8S5F3} with the statistical and systematic uncertainties given with the data points.

\section{Conclusions and Future Work}
\label{C8S6}

The goal of E08-027 is to obtain the proton spin structure function $g_2$ in the $Q^2$ range of 0.02-0.2 GeV${}^2$. In this chapter, the model prediction were used as input to extract $g_2$ since the acceptance study of E08-027 is not finalized. Once the acceptance study is finished, our data will be used to extract the unpolarized cross-sections for each kinematic setting in place of the P. Bosted's fits used here. In addition, the method for radiative correction in \Cref{C8S3} relies on the radiated cross-section models and thus is not accurate. This method will be updated with the standard deconvolution method.

From this preliminary extraction of the polarized cross-section differences, we could conclude that the data agrees with the model predictions. The MAID model and P. Bosted's fits agrees well in the region of the $\Delta$-resonance, however, the prediction for higher $W$ may not be as accurate. From \Cref{C8S2F2}, we noticed that the radiative effects are a large correction at high $W$, especially for the longitudinal configuration. This indicates that we should use our own data to extract the unpolarized cross-section and to perform the radiative correction, rather than relying on a cross-section model.

In \Cref{C8S4}, we also used the model predictions as input for $\Delta\sigma_{\parallel}$ since in most cases we only measured $\Delta\sigma_{\perp}$. The Jefferson Lab Hall B EG4 experiment measured $\Delta\sigma_{\parallel}$ in a similar kinematics range as this experiment. Thus, the model predictions of $\Delta\sigma_{\parallel}$ will be replaced by the data from EG4 once they finalized their analysis.

Once the studies mentioned above is done, the final results of the proton spin structure function $g_2$ will be extracted. These data will provide the first test of the BC sum rule at low $Q^2$, which has not been tested before. These data are also eagerly awaited to provide a benchmark test of the $\chi$PT predictions for the generalized spin polarizabilities $\gamma_0$ and $\dlt$.

%%%%%%%%%%%%%%%%%%%%%%%%%%%%%%%%%%%%%%%%%%%%%%%%%%%%%%%%%%%%%%%%%%%%%%
% -*-latex-*-
