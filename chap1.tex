% -*-latex-*-

%% This is an example first chapter.  You should put chapter/appendix that you
%% write into a separate file, and add a line \include{yourfilename} to
%% main.tex, where `yourfilename.tex' is the name of the chapter/appendix file.
%% You can process specific files by typing their names in at the
%% \files=
%% prompt when you run the file main.tex through LaTeX.

\chapter{Introduction}
\label{C1}

Nucleons, protons and neutrons, belong to the hadronic family of sub-atomic particles. The internal structore of the nucleon remained a mystery untill the 1960s. In the late 1960s, J. Friedman, H. Kendall and R. Taylor used a new high-energy electron beam at SLAC and found that the ratio of the differential cross-section and the Mott cross-section exhibits approximate scaling at large $Q^2$ \cite{Breidenbach1969}, which demonstrate the nucleon is composed of some point-like particles known as partons. We now know that nucleons are bound states of fundamental particles: quarks and gluons. The quarks interact with each other by exchanging gluons via the strong interaction. Quantum Chromodynamics (QCD) is the fundamental theory that describes the strong interaction. In the high energy region, QCD has been verified by numorous experimental results which have been compared to the perturbative solutions of the QCD Lagrangian. However, the non-perturbative confinement of quarks within the nucleon is still not well understood within QCD. This makes hadronic physics and the study of QCD is one of the most fascinating and challenging areas of modern science.

The spin structure of the nucleon has remained as one of the key issues of hadronic physics. Spin is one of the fundamental properties of particles. The investigation of spin began with the experiments of Stern and Gerlach in the early 1920s \cite{Gerlach1922}. Many decades later, the scattering experiments at powerful accelerators provide us the opportunity to begin to answer the question of how the quarks and gluons interact with each other to produce the spin of the nucleon. All the spin of the nucleon was expected to be carried by the quarks in the naive parton model. However, the first spin structure function experiments at SLAC \cite{Alguard1978} and CERN \cite{Ashman1988} showed that the total spin carried by quarks was very small. This puzzling result was known as the ``spin crisis''. Following this the spin structure of the nucleon became a highly productive area for both experiment and theory. During the past 30 years, many experiments were carried out to study the spin structure of the nucleon at SLAC, CERN, DESY, RHIC, Jefferson Lab (JLab) and other facilities \cite{Kuhn2009}. The purpose of these measurements was to examine how the total spin of the nucleon is distributed among its constituents. The present understanding is that the quarks only carries about 30\% of the total nucleon spin; the reset is expected to be carried by the quarks' orbital momentum and by the gluons. Although many experiment and theoretical efforts has been made, there are still many questions remaining along with some new challenges.

The structure of the nucleon is studied primarily through deep inelastic scattering (DIS) experiments which emphasize interaction with individual quarks and gluons at sufficiently high energies. In this asymptoticaly-free regime, the probes have afforded a great understanding of how the spin of the nucleon arises from its intrinsic degrees of freedom. However, physicists are not only focused on the DIS region where the internal interaction is relatively weak but also on the resonance region where the quarks and gluons are strongly interacting with each other. Results have become available recently from a new generation of JLab experiments which focused on probing the QCD in the non-perturbative and transition regimes. The different energy regimes provide probes with different resolutions, which could be used for complementary mapping of the strong interaction in the nucleon. The collective behavior of the nucleon constituents could also be retrieved from the low momentum transfer results in contrast to higher $Q^2$ where quark-gluon correlations are suppressed and parton-like behavior is observed \cite{G2P}.

The spin structure functions $g_1$ and $g_2$ of proton and neutron and their moments have been extracted over a wide kinematic range respectively \cite{Amarian2002,Amarian2004a,Amarian2004b,Wesselmann2007,Fatemi2003,Yun2003,Deur2004,Dharmawardane2004,Chen2004}. However, data on the spin structure function $g_2$ is absent for proton at low energy. Jefferson Lab Hall A Experiment E08-027 was carried out to provide precise data for proton $g_2$ in the low energy region to address intriguing discrepancies between neutron data and sum rule predictions from the chiral perturbation theory.

This thesis is organized as follows. \Cref{C2,C3} present the theory to formalize the inclusive scattering experiment. \Cref{C4} gives a review of the physics motivation behind E08-027. \Cref{C5} discusses the experimental setup at Jefferson Lab Hall A. \Cref{C6} gives a detailed discussion about spectrometer optics study and \Cref{C7} explains the rest of the data analysis. \Cref{C8} presents results and the conclusions.

%%%%%%%%%%%%%%%%%%%%%%%%%%%%%%%%%%%%%%%%%%%%%%%%%%%%%%%%%%%%%%%%%%%%%%
% -*-latex-*-
