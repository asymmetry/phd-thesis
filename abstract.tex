% -*-latex-*-
%
% $Log: abstract.tex,v $
% Revision 1.1  93/05/14  14:56:25  starflt
% Initial revision
%
% Revision 1.1  90/05/04  10:41:01  lwvanels
% Initial revision

%% The text of your abstract and nothing else (other than comments) goes here.
%% It will be single-spaced and the rest of the text that is supposed to go on
%% the abstract page will be generated by the abstractpage environment. This
%% file should be \input (not \include 'd) from cover.tex.

\noindent

The spin structure of the nucleon has remained as one of the key points of interest in hadronic physics, which has attracted many efforts from both experimentalists and theorists. Quantum Chromodynamics (QCD) is the fundamental theory that describes the strong interaction. It has been verified in the asymptotically free region. However, the non-perturbative confinement of quarks within the nucleon is still not well understood within QCD. In the non-perturbative regime, low-energy effective field theories such as chiral perturbation theory ($\chi$PT) provide predictions for the spin structure functions. The neutron spin structure functions, $g_1^n$ and $g_2^n$, and the proton spin structure function, $g_1^p$, have been measured over a wide kinematic range and compared with the theoretical predictions. However, the proton spin structure function, $g_2^p$, remains largely unmeasured.

The E08-027 collaboration successfully performed the first measurement of the inclusive electron-proton scattering in the kinematic range $0.02<Q^2<0.2$ GeV${}^2$. The experiment took place in experimental Hall A at Jefferson Lab in 2012. A longitudinally polarized electron beam with incident energies between 1.1 GeV and 3.3 GeV was scattered from a longitudinally or transversely polarized NH${}_3$ target. Asymmetries and polarized cross-section differences were measured in the resonance region to extract the proton spin structure functions $g_2$. The results allow us to obtain the generalized spin polarizabilities $\gamma_0$ and $\dlt$ and test the Burkhardtt-Cottingham (BC) sum rule. Chiral perturbation theory is expected to work in this kinematic range and this measurement of $\dlt$ will give a benchmark test to $\chi$PT calculations. This thesis will discuss preliminary results from the E08-027 data analysis.

%%%%%%%%%%%%%%%%%%%%%%%%%%%%%%%%%%%%%%%%%%%%%%%%%%%%%%%%%%%%%%%%%%%%%%
% -*-latex-*-
