% -*-latex-*-
%
% For questions, comments, concerns or complaints:
% thesis@mit.edu
%
%
% $Log: cover.tex,v $
% Revision 1.8  2008/05/13 15:02:15  jdreed
% Degree month is June, not May.  Added note about prevdegrees.
% Arthur Smith's title updated
%
% Revision 1.7  2001/02/08 18:53:16  boojum
% changed some \newpages to \cleardoublepages
%
% Revision 1.6  1999/10/21 14:49:31  boojum
% changed comment referring to documentstyle
%
% Revision 1.5  1999/10/21 14:39:04  boojum
% *** empty log message ***
%
% Revision 1.4  1997/04/18  17:54:10  othomas
% added page numbers on abstract and cover, and made 1 abstract
% page the default rather than 2.  (anne hunter tells me this
% is the new institute standard.)
%
% Revision 1.4  1997/04/18  17:54:10  othomas
% added page numbers on abstract and cover, and made 1 abstract
% page the default rather than 2.  (anne hunter tells me this
% is the new institute standard.)
%
% Revision 1.3  93/05/17  17:06:29  starflt
% Added acknowledgements section (suggested by tompalka)
%
% Revision 1.2  92/04/22  13:13:13  epeisach
% Fixes for 1991 course 6 requirements
% Phrase "and to grant others the right to do so" has been added to
% permission clause
% Second copy of abstract is not counted as separate pages so numbering works
% out
%
% Revision 1.1  92/04/22  13:08:20  epeisach

% NOTE:
% These templates make an effort to conform to the MIT Thesis specifications,
% however the specifications can change. We recommend that you verify the
% layout of your title page with your thesis advisor and/or the MIT
% Libraries before printing your final copy.

\title{The Spin Structure of the Proton at Low $Q^2$: A Measurement of the Structure Function $g_2^p$}

\makeatletter
\author{Chao Gu} \let\Author\@author
% If you wish to list your previous degrees on the cover page, use the
% previous degrees command:
%       \prevdegrees{A.A., Harvard University (1985)}
% You can use the \\ command to list multiple previous degrees
%       \prevdegrees{B.S., University of California (1978) \\
%                    S.M., Massachusetts Institute of Technology (1981)}
\prevdegrees{Shiyan, Hubei, China \\
             B.S. Peking University, 2009}
\department{Department of Physics}

% If the thesis is for two degrees simultaneously, list them both
% separated by \and like this:
% \degree{Doctor of Philosophy \and Master of Science}
\degree{Doctor of Philosophy}

% As of the 2007-08 academic year, valid degree months are September,
% February, or June.  The default is June.
\degreemonth{June}
\degreeyear{2016} \let\Year\@degreeyear
\thesisdate{June 6, 2016}
\makeatother

%% By default, the thesis will be copyrighted to MIT.  If you need to copyright
%% the thesis to yourself, just specify the `vi' documentclass option.  If for
%% some reason you want to exactly specify the copyright notice text, you can
%% use the \copyrightnoticetext command.
%\copyrightnoticetext{\copyright IBM, 1990.  Do not open till Xmas.}

% If there is more than one supervisor, use the \supervisor command
% once for each.
\supervisor{N. K. Liyanage}{Professor}

% This is the department committee chairman, not the thesis committee
% chairman.  You should replace this with your Department's Committee
% Chairman.
\chairman{N/A}{N/A}

% Make the titlepage based on the above information.  If you need
% something special and can't use the standard form, you can specify
% the exact text of the titlepage yourself.  Put it in a titlepage
% environment and leave blank lines where you want vertical space.
% The spaces will be adjusted to fill the entire page.  The dotted
% lines for the signatures are made with the \signature command.
\makeatletter
\def\maketitle{\begin{titlepage}
\doublespacing
\vspace{0.8in}
\large
{\LARGE\bf \@title \par}
\@author \\
\@prevdegrees
\par
A Dissertation Presented to the Graduate Faculty \\
of the University of Virginia in Candidacy for the Degree of \\
\@degree
\par
\@department
\par
University of Virginia \\
\@degreemonth, \@degreeyear
\vspace{0.8in}
\end{titlepage}}
\makeatother
\maketitle

% The abstractpage environment sets up everything on the page except
% the text itself.  The title and other header material are put at the
% top of the page, and the supervisors are listed at the bottom.  A
% new page is begun both before and after.  Of course, an abstract may
% be more than one page itself.  If you need more control over the
% format of the page, you can use the abstract environment, which puts
% the word "Abstract" at the beginning and single spaces its text.

%% You can either \input (*not* \include) your abstract file, or you can put
%% the text of the abstract directly between the \begin{abstractpage} and
%% \end{abstractpage} commands.

% First copy: start a new page, and save the page number.
% \cleardoublepage
% Uncomment the next line if you do NOT want a page number on your
% abstract and acknowledgments pages.
% \pagestyle{empty}
% \setcounter{savepage}{\thepage}
% \begin{abstractpage}
% % -*-latex-*-
%
% $Log: abstract.tex,v $
% Revision 1.1  93/05/14  14:56:25  starflt
% Initial revision
%
% Revision 1.1  90/05/04  10:41:01  lwvanels
% Initial revision

%% The text of your abstract and nothing else (other than comments) goes here.
%% It will be single-spaced and the rest of the text that is supposed to go on
%% the abstract page will be generated by the abstractpage environment. This
%% file should be \input (not \include 'd) from cover.tex.

\noindent

The spin structure of the nucleon has remained as one of the key points of interest in hadronic physics, which has attracted many efforts from both experimentalists and theorists. Quantum Chromodynamics (QCD) is the fundamental theory that describes the strong interaction. It has been verified in the asymptotically free region. However, the non-perturbative confinement of quarks within the nucleon is still not well understood within QCD. In the non-perturbative regime, low-energy effective field theories such as chiral perturbation theory ($\chi$PT) provide predictions for the spin structure functions. The neutron spin structure functions, $g_1^n$ and $g_2^n$, and the proton spin structure function, $g_1^p$, have been measured over a wide kinematic range and compared with the theoretical predictions. However, the proton spin structure function, $g_2^p$, remains largely unmeasured.

The E08-027 collaboration successfully performed the first measurement of the inclusive electron-proton scattering in the kinematic range $0.02<Q^2<0.2$ GeV${}^2$. The experiment took place in experimental Hall A at Jefferson Lab in 2012. A longitudinally polarized electron beam with incident energies between 1.1 GeV and 3.3 GeV was scattered from a longitudinally or transversely polarized NH${}_3$ target. Asymmetries and polarized cross-section differences were measured in the resonance region to extract the proton spin structure functions $g_2$. The results allow us to obtain the generalized spin polarizabilities $\gamma_0$ and $\dlt$ and test the Burkhardtt-Cottingham (BC) sum rule. Chiral perturbation theory is expected to work in this kinematic range and this measurement of $\dlt$ will give a benchmark test to $\chi$PT calculations. This thesis will discuss preliminary results from the E08-027 data analysis.

%%%%%%%%%%%%%%%%%%%%%%%%%%%%%%%%%%%%%%%%%%%%%%%%%%%%%%%%%%%%%%%%%%%%%%
% -*-latex-*-

% \end{abstractpage}

% Additional copy: start a new page, and reset the page number.  This way,
% the second copy of the abstract is not counted as separate pages.
% Uncomment the next 6 lines if you need two copies of the abstract
% page.
% \setcounter{page}{\thesavepage}
% \begin{abstractpage}
% % -*-latex-*-
%
% $Log: abstract.tex,v $
% Revision 1.1  93/05/14  14:56:25  starflt
% Initial revision
%
% Revision 1.1  90/05/04  10:41:01  lwvanels
% Initial revision

%% The text of your abstract and nothing else (other than comments) goes here.
%% It will be single-spaced and the rest of the text that is supposed to go on
%% the abstract page will be generated by the abstractpage environment. This
%% file should be \input (not \include 'd) from cover.tex.

\noindent

The spin structure of the nucleon has remained as one of the key points of interest in hadronic physics, which has attracted many efforts from both experimentalists and theorists. Quantum Chromodynamics (QCD) is the fundamental theory that describes the strong interaction. It has been verified in the asymptotically free region. However, the non-perturbative confinement of quarks within the nucleon is still not well understood within QCD. In the non-perturbative regime, low-energy effective field theories such as chiral perturbation theory ($\chi$PT) provide predictions for the spin structure functions. The neutron spin structure functions, $g_1^n$ and $g_2^n$, and the proton spin structure function, $g_1^p$, have been measured over a wide kinematic range and compared with the theoretical predictions. However, the proton spin structure function, $g_2^p$, remains largely unmeasured.

The E08-027 collaboration successfully performed the first measurement of the inclusive electron-proton scattering in the kinematic range $0.02<Q^2<0.2$ GeV${}^2$. The experiment took place in experimental Hall A at Jefferson Lab in 2012. A longitudinally polarized electron beam with incident energies between 1.1 GeV and 3.3 GeV was scattered from a longitudinally or transversely polarized NH${}_3$ target. Asymmetries and polarized cross-section differences were measured in the resonance region to extract the proton spin structure functions $g_2$. The results allow us to obtain the generalized spin polarizabilities $\gamma_0$ and $\dlt$ and test the Burkhardtt-Cottingham (BC) sum rule. Chiral perturbation theory is expected to work in this kinematic range and this measurement of $\dlt$ will give a benchmark test to $\chi$PT calculations. This thesis will discuss preliminary results from the E08-027 data analysis.

%%%%%%%%%%%%%%%%%%%%%%%%%%%%%%%%%%%%%%%%%%%%%%%%%%%%%%%%%%%%%%%%%%%%%%
% -*-latex-*-

% \end{abstractpage}

% Copyright page
\pagestyle{empty}
\newpage
\vspace*{\fill}
\noindent \textcopyright Copyright by \Author {} \Year \\
All Rights Reserved

\cleardoublepage

% Abstract page
\vspace{0.8in}
\pdfbookmark[0]{Abstract}{Abstract}
\section*{\center Abstract}
% -*-latex-*-
%
% $Log: abstract.tex,v $
% Revision 1.1  93/05/14  14:56:25  starflt
% Initial revision
%
% Revision 1.1  90/05/04  10:41:01  lwvanels
% Initial revision

%% The text of your abstract and nothing else (other than comments) goes here.
%% It will be single-spaced and the rest of the text that is supposed to go on
%% the abstract page will be generated by the abstractpage environment. This
%% file should be \input (not \include 'd) from cover.tex.

\noindent

The spin structure of the nucleon has remained as one of the key points of interest in hadronic physics, which has attracted many efforts from both experimentalists and theorists. Quantum Chromodynamics (QCD) is the fundamental theory that describes the strong interaction. It has been verified in the asymptotically free region. However, the non-perturbative confinement of quarks within the nucleon is still not well understood within QCD. In the non-perturbative regime, low-energy effective field theories such as chiral perturbation theory ($\chi$PT) provide predictions for the spin structure functions. The neutron spin structure functions, $g_1^n$ and $g_2^n$, and the proton spin structure function, $g_1^p$, have been measured over a wide kinematic range and compared with the theoretical predictions. However, the proton spin structure function, $g_2^p$, remains largely unmeasured.

The E08-027 collaboration successfully performed the first measurement of the inclusive electron-proton scattering in the kinematic range $0.02<Q^2<0.2$ GeV${}^2$. The experiment took place in experimental Hall A at Jefferson Lab in 2012. A longitudinally polarized electron beam with incident energies between 1.1 GeV and 3.3 GeV was scattered from a longitudinally or transversely polarized NH${}_3$ target. Asymmetries and polarized cross-section differences were measured in the resonance region to extract the proton spin structure functions $g_2$. The results allow us to obtain the generalized spin polarizabilities $\gamma_0$ and $\dlt$ and test the Burkhardtt-Cottingham (BC) sum rule. Chiral perturbation theory is expected to work in this kinematic range and this measurement of $\dlt$ will give a benchmark test to $\chi$PT calculations. This thesis will discuss preliminary results from the E08-027 data analysis.

%%%%%%%%%%%%%%%%%%%%%%%%%%%%%%%%%%%%%%%%%%%%%%%%%%%%%%%%%%%%%%%%%%%%%%
% -*-latex-*-


\cleardoublepage

% Acknowledgemnt page
%\pagenumbering{roman}
%\pagestyle{plain}
%\pdfbookmark[0]{Acknowledgments}{Acknowledgments}
%\section*{Acknowledgments}
%% -*-latex-*-

%% The text of your acknowledgement and nothing else (other than comments)
%% goes here. It will be single-spaced and the rest of the text that is
%% supposed to go on the acknowledgement page will be generated by the
%% abstractpage environment. This file should be \input (not \include 'd)
%% from cover.tex.

There have been so many people who provided support and encouragement to me during this long journey to complete this dissertation. I would like to deeply thank everyone providing me with so much help.

First, I would like to thank my thesis advisor, Nilanga Liyanage, who has been such a terrific mentor and friend during the past years. I would not have reach this point without his help and guidiance. My discussions with him were always enjoyable, not only enhanced my physics insight, but also helped me to become independent and to understand how to become a physicist.

I would like to thank my Jefferson Lab supervisor, Jian-ping Chen, who quickly brought me involved in the collabration and enthusiastically answered my questions and help me to better understand the physics behind the experiment. I spent most of my graduate time at Jefferson Lab and I learned a lot from his advices and help, both in research and in everyday life. It was my great privilege to get this chance to work with him.

I would like to thank my thesis committee members: Edward Murphy, Kent Paschke and Xiaochao Zheng for their time in carefully going through this document and providing valuable comments.

I would like to thank Alexandre Camsonne, Jian-Ping Chen, Don Crabb and Karl Slifer for their role as the spokepersons in this experiment. Our experiment experienced many problems. Their delibrate plan guaranteed a promising experiment. I would also like to thank the Hall A collaboration and the JLab target group. Without their efforts, the experiment would not have been a success.

I would like to thank my colleagues, either graduate students or post-docs, for their hard work to ensure the experiment was a success: Toby Badman, Melissa Cummings, Min Huang, Jie Liu, Pengjia Zhu, Ryan Zielinski, Kalyan Allada, Ellie Long, James Maxwell, Vince Sulkosky and Jixie Zhang. Thanks to Jixie for teaching me to build a simulation package and for many valuable discussions on the spectrometer optics study. I would also like to thank Kalyan for teaching me the detector and DAQ setup.

I would like to thank all my friends at University of Virginia and Jefferson Lab. Their friendship made my graduate study an enjoyable experience. They are: Xiaping Tang, Zukai Wang, Yuxiang Zhao, Li Ye, Xiuchang Yang, Yi Qiang, Chaolun Wu, Chunhua Chen, Yuncheng Han, Zhihong Ye, Zongwen Yang, Longwu Ou, Yang Wang, Zhiwen Zhao, He Zhang, He Huang, Kai Jin, Peng Peng, Wei Liu, Jin Huang, Chao Peng, Yawei Zhang, Wei Tang, Siyu Jian, Danning Di, Kondo Gnanvo, Aiwu Zhang, Mei Zhang, Yifei Shi, Luna Yang, Moran Chen, Ryan Duve.

I would like to thank my parents for their persistent support and endless encouragements. It was not easy for them to push me so far away to pursue my dream, but they were always understanding of my decision.

Finally, thanks to my wife Deli Zhu for all her love and support which made all of this possible.

%%%%%%%%%%%%%%%%%%%%%%%%%%%%%%%%%%%%%%%%%%%%%%%%%%%%%%%%%%%%%%%%%%%%%%
% -*-latex-*-


%%%%%%%%%%%%%%%%%%%%%%%%%%%%%%%%%%%%%%%%%%%%%%%%%%%%%%%%%%%%%%%%%%%%%%
% -*-latex-*-
